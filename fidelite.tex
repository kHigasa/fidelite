\documentclass[b5paper,12pt]{tarticle}

\begin{document}
\tableofcontents
\section*{二〇二〇}
\subsection*{自己啓発本の類}
読書好きと言うので何を読むのかと聞くと、「自己啓発本」と言う人が居る。こちらとしては文学や理学等、そういったジャンルの中の何らかのカテゴリが返ってくるのだろうとばかり思っているので、こう言われるとあっけらかんとしてしまう。

そもそも自己啓発本などというものは、書籍の装丁を施されてはいるものの、その実、物理的な紙面と著者名のインパクトヴァリューに金銭的価値を付しただけの計算用紙に変わりない(それにしてはかなり高いようだが())。

そのためアレが書物の筈はない。

しかしながら世の中にはあれが読書の対象だと言われても、何ら違和感を感じない人々もあるらしい。普通に怖い。

%\subsubsection*{          *}

\section*{二〇二一}
\subsection*{永井荷風『東綺譚』勘違い恋愛小説}
『東綺譚』は,「玉の井の私娼街を背景として人事に添えて夏から秋への季節の移り
ゆくさまを描写」(p. 116)した風俗小説にして,作者の意図として「玉の井という昭和の私娼窟を
風物詩的に後世に伝え残そうとした」(p. 117),「作中人物の生活や事件が展開する場所や背景を
情味を以て克明に描き写した一種の随筆的小説」(p. 116。引用はいずれも新潮文庫版
『東綺譚』の秋庭太郎による解説)と評される。
\end{document}
